% Created 2022-02-04 Fri 11:48
% Intended LaTeX compiler: pdflatex
\documentclass[12pt, reqno, oneside]{amsbook}
              \usepackage[framemethod=TikZ, skipabove=10pt, skipbelow=10pt, backgroundcolor=black!3, roundcorner=4pt, linewidth=1pt]{mdframed}
\BeforeBeginEnvironment{minted}{\begin{mdframed}}
\AfterEndEnvironment{minted}{\end{mdframed}}
\usepackage{appendix}
\usepackage{url}


\usepackage[utf8]{inputenc}
\usepackage[T1]{fontenc}
\usepackage{graphicx}
\usepackage{grffile}
\usepackage{longtable}
\usepackage{wrapfig}
\usepackage{rotating}
\usepackage[normalem]{ulem}
\usepackage{amsmath}
\usepackage{textcomp}
\usepackage{amssymb}
\usepackage{capt-of}
\usepackage{hyperref}
\usepackage[newfloat]{minted}
\usepackage{caption}
\author{Yi Zhang}
\date{\today}
\title{\texttt{Fazang}: A Reverse-mode Automatic differentiation tool in Fortran\\\medskip
\large User's Guide \\  (Version 0.0.1)}
\hypersetup{
 pdfauthor={Yi Zhang},
 pdftitle={\texttt{Fazang}: A Reverse-mode Automatic differentiation tool in Fortran},
 pdfkeywords={},
 pdfsubject={},
 pdfcreator={Emacs 27.2 (Org mode 9.4.4)}, 
 pdflang={English}}
\begin{document}

\maketitle
\tableofcontents


\chapter{Introduction}
\label{sec:org58c8df1}
\texttt{Fazang} is a reverse-mode automatic differentiation (AD) tool. The
project is heavily influenced by \texttt{Stan/Math} \cite{Carpenter:2015}, a project the author
is also involved in. \texttt{Fazang} is intended to support general scientific
computing in Fortran beyond Bayesian inference and Markove Chain
Monte Carlo that \texttt{Stan/Math} is designed for. 

User should be aware that the project is at early stage and still
under development. For any questions, suggestions, and
contributions, please visit the project at \url{https://github.com/yizhang-yiz/fazang}.
\chapter{Quick Start}
\label{sec:org7e86c8b}
Currently \texttt{Fazang} has been tested on Linux and MacOS platform, with
Fortran compiler Intel Fortran 19.0.1+ and GNU Fortran 11.2.0+.

After downloading \texttt{Fazang}, user can use \texttt{meson} to build the library.
\begin{minted}[breaklines=true,fontsize=\footnotesize,breakanywhere=true]{bash}
git clone git@github.com:yizhang-yiz/fazang.git
cd fazang && mkdir build && cd build
meson compile
\end{minted}
This generates a shared library at \texttt{build/src/}. User needs to link
this library when building an application. This can be done in
\texttt{meson} by setting
\begin{minted}[breaklines=true,fontsize=\footnotesize,breakanywhere=true]{python}
executable('app_name', files('path/to/app_file.F90'), dependencies : fazang_dep)
\end{minted}

\texttt{Fazang} provides a user-facing derived type \texttt{var}. This is the
type for the dependent and independent variables of which the
adjoint (derivative) will be calculated.

For example, consider the log of
the Gaussian distribution density with mean \(\mu\) and
standard deviation \(\sigma\)
\begin{equation}\label{eq:lnormal_example}
  f(\mu, \sigma) = \log{\left(
      \frac{1}{\sigma\sqrt{2\pi}} \exp\left(
        -\frac{1}{2}\left(\frac{y-\mu}{\sigma}\right)^2
      \right)
    \right)}
\end{equation}
The following programe calculates \(\frac{df}{d\mu}\) and
      \(\frac{df}{d\sigma}\) at \(y=1.3\), \(\mu=0.5\), and \(\sigma=1.2\).
\begin{minted}[breaklines=true,fontsize=\footnotesize,breakanywhere=true]{fortran}
program log_demo
  use fazang ! load Fazang library

  implicit none

  real(rk) :: y
  type(var) :: f, sigma, mu

  ! data
  y = 1.3d0

  ! independent variables
  mu = var(0.5d0)
  sigma = var(1.2d0)

  ! dependent
  f = var(-0.5d0 * log(2 * pi))
  f = f - log(sigma)
  f = f - 0.5d0 * ((y - mu) / sigma) ** 2.d0;

  ! use grad() to calculate df/d(mu) and df/d(sigma). Each var's
  ! derivative (also called adjoint) can be access through var%adj().

  call f%grad()
  write(*, *) "df/d(mu): ", mu%adj()
  write(*, *) "df/d(sigma): ", sigma%adj()
end program log_demo
\end{minted}

\chapter{Use \texttt{Fazang}}
\label{sec:org71f919e}
\texttt{Fazang} uses \texttt{var} type to record numerical and gradient
operations. The type supports three functions
\begin{itemize}
\item \mintinline[breaklines=true,fontsize=\footnotesize,breakanywhere=true]{fortran}{var%val()} : returns value
\item \mintinline[breaklines=true,fontsize=\footnotesize,breakanywhere=true]{fortran}{var%adj()} : returns derivative, henceforth referred as \emph{adjoint}.
\item \mintinline[breaklines=true,fontsize=\footnotesize,breakanywhere=true]{fortran}{var%grad()} : takes gradient operation with respect to the current \mintinline[breaklines=true,fontsize=\footnotesize,breakanywhere=true]{fortran}{var} variable.
\end{itemize}
\section{Constructors}
\label{sec:orge3eb514}
\texttt{var} can be constructed using overloaded \texttt{var} interface.
\begin{minted}[breaklines=true,fontsize=\footnotesize,breakanywhere=true]{fortran}
real(real64) :: a, b(3), c(2, 3)
real(real64) :: new_a, new_b(3), new_c(2, 3)
type(var) :: x, y(3), z(2, 3)
! ...
x = var()                 ! x%val() == 0.d0
x = var(a)                ! x%val() == a
y = var(b)                ! y%val() == b
z = var(c)                ! z%val() == c
\end{minted}
\section{Assignment}
\label{sec:org9a6fc87}
\texttt{var} can be assigned from consistent \texttt{var} and \texttt{real(real64)}.
\begin{minted}[breaklines=true,fontsize=\footnotesize,breakanywhere=true]{fortran}
! ....
x = new_a                 ! x%val() == new_a
y = new_b                 ! y%val() == new_b
z = new_c                 ! z%val() == new_c
\end{minted}

\section{Gradient}
\label{sec:org17a2cbe}
   \label{sec:gradient}
Consider a variable \(z\) calculated by the composition of a series of operations
\begin{equation*}
z = f_1(z_1), \quad z_1 = f_2(z_2), \quad \dots, \quad z_{n-1} = f_n(z_n).
\end{equation*}
For \(z_i, i = 1, \dots, n\) we refer \(dz/d{z_i}\) as the \emph{adjoint} of \(z_i\),
denoted by \(z_i^{\text{adj}}\).
The chain rule says the adjoints can be calculated recursively \cite{griewank_evaluating_2008},
\begin{align*}
z^{\text{adj}} = 1, \\
z_1^{\text{adj}} = z^{\text{adj}} \frac{df_1}{dz_1}, \\
\dots, \\
z_i^{\text{adj}} = z_{i-1}^{\text{adj}} \frac{df_i}{dz_i}.
\end{align*}

We often refer each \((f_i, z_i)\) pair as a
\emph{node}, and \(z_i\) the \emph{operand} of operation \(f_i\). The above recursion through the nodes requires a way to store
and visit the \emph{callstack} of nodes.  It is embodied in \texttt{Fazang} by the \texttt{var\%grad()}
function. When \texttt{z\%grad()} is called, \texttt{z}'s adjoint is set to 1, and
every other \texttt{var} variable is transversed with its adjoint updated. In
order to calculate the adjoint with respect to another variable, user
must \texttt{call set\_zero\_all\_adj()} first to reset all adjoints to zero.

An alternative to invoke gradient calculation is to define the
dependent as a function and feed it to \texttt{Fazang}'s \texttt{gradient}
function. Take Eq.\eqref{eq:lnormal_example} for example, we can first
define the function for \(f(\mu, \sigma)\).
\begin{minted}[breaklines=true,fontsize=\footnotesize,breakanywhere=true]{fortran}
module func
  use fazang ! load Fazang library
  implicit none

  real(rk), parameter :: y = 1.3d0

contains
  type(var) function f(x)
    type(var), intent(in) :: x(:)
    type(var) :: mu, sigma
    mu = x(1)
    sigma = x(2)
    f = -0.5d0 * log(2 * pi) - log(sigma) - 0.5d0 * ((y - mu) / sigma) ** 2.d0;
  end function f

end module func
\end{minted}
Then we can supply function \texttt{f} as a procedure argument.
\begin{minted}[breaklines=true,fontsize=\footnotesize,breakanywhere=true]{fortran}
program log_demo2
  use iso_c_binding
  use fazang
  use func

  implicit none

  real(real64) :: fx(3), x(2)
  x = [0.5d0, 1.2d0]

  fx = gradient(f, x)
  write(*, *) "f(x): ", fx(1)
  write(*, *) "df/d(x(1)): ", fx(2)
  write(*, *) "df/d(x(2)): ", fx(3)
end program log_demo2
\end{minted}
The output of \texttt{gradient(f, x)} is an array of size \texttt{1 + size(x)}, with
first component being the function value, and the rest the partial
derivatives.

\section{Functions}
\label{sec:orgcdd6c76}
Numeric functions supported by \texttt{Fazang} are listed in Appendix \ref{appendix:func}. All unary and
binary functions are \texttt{elemental}. The binary functions allow mixed
argument types, namely, either argument can be \texttt{real64} type while the
other the \texttt{var} type.

\chapter{Design}
\label{sec:org66faff8}
The core of any reverse-mode automatic differentiation is the data
structure to store and visit the callstack. \texttt{Fazang} achieves this
through two derived types, \texttt{tape} and \texttt{vari}.

\section{\texttt{tape} data structure}
\label{sec:org9b4eff5}
A \texttt{tape} is an \texttt{int32} array emulating a stack, with an integer marker \texttt{head} pointing to the
head to the current stack top.
\begin{minted}[breaklines=true,fontsize=\footnotesize,breakanywhere=true]{fortran}
type :: tape
     integer(ik) :: head = 1
     integer(ik), allocatable :: storage(:)
!...
\end{minted}
Each time a new AD node is created,
space in \texttt{storage} is allotted to store the node's
\begin{itemize}
\item value \(f_i(z_i)\),
\item adjoint \(z_{i-1}^{\text{adj}}\),
\item number of \texttt{var} operands of \(f_i\),
\item The \texttt{var} operands' index in the same \texttt{tape} array,
\item number of \texttt{real64} operands of \(f_i\),
\item The \texttt{real64} operands' value.
\end{itemize}

Since a node's value, adjoint, and data
operands are \texttt{real64}, they are first converted to \texttt{int32} using
\texttt{transfer} function before stored in the \texttt{tape} array, so that each such
a value occupies two \texttt{storage} entries. After each
allotation, the \texttt{head} is moved to point to the next empty slot in
the array after saving its current value to a \texttt{vari} type variable
for future retrieval.

\section{\texttt{vari} type}
\label{sec:org419f147}
The \texttt{vari} type is simply a proxy of a node's storage location in the tape
\begin{minted}[breaklines=true,fontsize=\footnotesize,breakanywhere=true]{fortran}
type :: vari
  integer(ik) :: i = 0
  procedure(chain_op), pass, pointer :: chain
contains
   !....
\end{minted}
where \texttt{i} is the index to the beginning of a node's storage, and the
\texttt{chain} procedure encodes the node's operation
\(f_i\). \texttt{chain} follows an interface that describes the chain rule
operation
\begin{minted}[breaklines=true,fontsize=\footnotesize,breakanywhere=true]{fortran}
abstract interface
   subroutine chain_op(this)
     import :: vari
     class(vari), intent(in) :: this
   end subroutine chain_op
end interface
\end{minted}
An alternative to integer index is to a \texttt{pointer} to the according
enry in the \texttt{tape} array. However, we will need to expand the
\texttt{storage} when it is filled up, and \texttt{Fazang} does this by doubling the
\texttt{storage} size and use \texttt{move\_alloc} to
restore the original values. Since there is no guarantee that \texttt{move\_alloc}
will keep the original memory, a pointer to the original address would
be corrupted.

As a \texttt{Fazang} program steps forward, a series of \texttt{vari} variables are
generated, with their \emph{values} calculated and stored. This is called
a \emph{forward pass}. The generated \texttt{vari} variables in the forward pass are
stored in array \texttt{varis}. Each entry in \texttt{varis} is a dependent
(operation output) of one or more previous entries.

\section{\texttt{var} type}
\label{sec:orgff9d04e}
The user-facing \texttt{var} type serves as proxy to \texttt{vari}. Each \texttt{var}
stores the index of a \texttt{vari} in the \texttt{varis} array.
\begin{minted}[breaklines=true,fontsize=\footnotesize,breakanywhere=true]{fortran}
type :: var
   integer(int32) :: vi
 contains
   procedure :: val
   procedure :: adj
   procedure :: grad
   procedure :: set_chain
end type var
\end{minted}
After the forward pass, when adjoints are desired, we call \texttt{grad} or
\texttt{gradient} procedure. This initiates a \emph{reverse pass}, in which  the
\texttt{varis} array is transversed backward
so that each \texttt{vari}'s \texttt{chain} procedure is called to update the
operand adjoints.
\begin{minted}[breaklines=true,fontsize=\footnotesize,breakanywhere=true]{fortran}
subroutine grad(this)
  class(var), intent(in) :: this
  integer i
  call callstack % varis (this%vi) % init_dependent()
  do i = callstack % head - 1, 1, -1
     call callstack % varis(i) % chain()
  end do
end subroutine grad
\end{minted}
Here \texttt{callstack} is the module variable that encapsulate \texttt{tape} and
\texttt{varis} arrays.

\chapter{Add operation functions}
\label{sec:orgb738ad9}
Adding an operation \(f_i\) involves creating functions for forward
pass and reverse pass. Let us first use \texttt{log} function as a simple
example.

First, we create a \texttt{log\_v} function for the forward pass.
\begin{minted}[breaklines=true,fontsize=\footnotesize,breakanywhere=true]{fortran}
impure elemental function log_v(v) result(s)
  type(var), intent(in) :: v
  type(var) :: s
  s = var(log(v%val()), [v])
  call s%set_chain(chain_log)
end function log_v
\end{minted}
The function generates a new \texttt{var} variable \texttt{s} using a special
constructor \texttt{var(value, array of operands)} which stores the value as
well as the single operand \texttt{v}'s index (in the \texttt{tape} \texttt{storage}
array). It also points \texttt{s}'s chain to a dedicated procedure \texttt{chain\_log}.
\begin{minted}[breaklines=true,fontsize=\footnotesize,breakanywhere=true]{fortran}
subroutine chain_log(this)
  class(vari), intent(in) :: this
  real(rk) :: new_adj(1), val(1)
  new_adj = this%operand_adj()
  val = this%operand_val()
  new_adj(1) = new_adj(1) + this%adj() / val(1)
  call this%set_operand_adj(new_adj)
end subroutine chain_log
\end{minted}
To understand this function, recall the recursion in Section \ref{sec:gradient},
assume the \texttt{log} operation is node \(i\), then \(f_i=\log(\dot)\) and
\(z_i\) is the operand \texttt{v}, and the new \texttt{var} \texttt{s} would be
\(z_{i-1}\). During the reverse pass when the node is visited, \texttt{chain\_log} 
first retrieves current \((z_i, z_i^{\text{adj}})\)
using \texttt{operand\_val()} and \texttt{operand\_adj()}, then updates
\(z_i^{\text{adj}}\) with an additional
\begin{equation*}
z_{i-1}^{\text{adj}} \frac{df_i}{dz_i} = z_{i-1}^{\text{adj}}\frac{d\log(z_i)}{dz_i}=\frac{z_{i-1}^{\text{adj}}}{z_i}.
\end{equation*}

Adding a binary operation \(f_i(z_i^{(1)}, z_i^{2})\) is slightly more complex, as we will need to
address possibly different scenarios when \(z_i^{(1)}\) and \(z_i^{(2)}\)
are either \texttt{var} or \texttt{real64}. Let us use overloaded division \texttt{operator(/)} as an example.

With
\begin{equation*}
f_i(z_i^{(1)}, z_i^{2}) = z_i^{(1)} / z_i^{(2)}
\end{equation*}
we need to account for
\begin{itemize}
\item both \(z_i^{(1)}\) and \(z_i^{2}\) are \texttt{var}'s
\item \(z_i^{(1)}\) is \texttt{var}, \(z_i^{2}\) is \texttt{real64},
\item \(z_i^{(1)}\) is \texttt{real64}, \(z_i^{2}\) is \texttt{var},
\end{itemize}


For the first scenario, we create
\begin{minted}[breaklines=true,fontsize=\footnotesize,breakanywhere=true]{fortran}
impure elemental function div_vv(v1, v2) result(s)
  type(var), intent(in) :: v1, v2
  type(var) :: s
  s = var(v1%val() / v2%val(), [v1, v2])
  call s%set_chain(chain_div_vv)
end function div_vv
\end{minted}
Similar to the \texttt{log} example, we create a new \texttt{s} with both operands
stored. In the corresponding \texttt{chain} procedure, we need update
the adjoints of both \texttt{v1} and \texttt{v2}.
\begin{minted}[breaklines=true,fontsize=\footnotesize,breakanywhere=true]{fortran}
subroutine chain_div_vv(this)
  class(vari), intent(in) :: this
  real(rk) :: new_adj(2), val(2)
  new_adj = this%operand_adj()
  val = this%operand_val()
  new_adj(1) = new_adj(1) + this%adj()/val(2)
  new_adj(2) = new_adj(2) - this%val() * this%adj()/val(2)
  call this%set_operand_adj(new_adj)
end subroutine chain_div_vv
\end{minted}

For the second scenario, we create
\begin{minted}[breaklines=true,fontsize=\footnotesize,breakanywhere=true]{fortran}
impure elemental function div_vd(v, d) result(s)
  type(var), intent(in) :: v
  real(rk), intent(in) :: d
  type(var) :: s
  s = var(v%val() / d, [v], [d])
  call s%set_chain(chain_div_vd)
end function div_vd
\end{minted}
Again we create a new \texttt{var} \texttt{s}. But this time
we use another constructor \texttt{var(value, var operands, data
operands)} to store value, \texttt{var} operand \texttt{v}, and \texttt{real64}
operand \texttt{d}. In the corresponding reverse pass \texttt{chain} procedure, not
      only we need retrieve \texttt{var} operand \texttt{v} but also data operand
      \texttt{d}, as the new adjoint of \(z_i^{(1)}\) is
\begin{equation*}
z_i^{(1)\text{new adj}} = z_i^{(1)\text{old adj}} + z_{i-1}^{\text{adj}}\frac{df_i}{dz_i^{(1)}}
= z_i^{(1)\text{old adj}} + z_{i-1}^{\text{adj}}\frac{1}{dz_i^{(2)}}
\end{equation*}      
So with \texttt{v} as \(z_i^{(1)}\) and \texttt{d} as \(z_i^{(2)}\) we have
\begin{minted}[breaklines=true,fontsize=\footnotesize,breakanywhere=true]{fortran}
subroutine chain_div_vd(this)
  class(vari), intent(in) :: this
  real(rk) d(1), new_adj(1)
  new_adj = this%operand_adj()
  d = this%data_operand()
  new_adj(1) = new_adj(1) + this%adj() / d(1)
  call this%set_operand_adj(new_adj)
end subroutine chain_div_vd
\end{minted}

The third scenario is treated similarly.

\appendix
\chapter{\texttt{Fazang} Functions \label{sec:func_list}}
\label{sec:org63ffb65}
\label{appendix:func}  
\begin{center}
\begin{tabular}{lll}
Function & Argument(s) & Operation\\
\hline
\texttt{sin} & scalar or array & same as intrinsic\\
\texttt{cos} & scalar or array & same as intrinsic\\
\texttt{tan} & scalar or array & same as intrinsic\\
\texttt{asin} & scalar or array & same as intrinsic\\
\texttt{acos} & scalar or array & same as intrinsic\\
\texttt{atan} & scalar or array & same as intrinsic\\
\texttt{log} & scalar or array & same as intrinsic\\
\texttt{exp} & scalar or array & same as intrinsic\\
\texttt{squrt} & scalar or array & same as intrinsic\\
\texttt{square} & scalar or array & For input \texttt{x}, calculate \texttt{x**2}\\
\texttt{inv} & scalar or array & For input \texttt{x}, calculate \texttt{1/x}\\
\texttt{inv\_square} & scalar or array & For input \texttt{x}, calculate \texttt{1/x**2}\\
\texttt{inv\_sqrt} & scalar or array & For input \texttt{x}, calculate \texttt{1/sqrt(x)}\\
\texttt{logit} & scalar or array & For input \texttt{x}, calculate \texttt{log(x/(1-x))}\\
\texttt{inv\_logit} & scalar or array & For input \texttt{x}, calculate \texttt{1/(1+exp(-x))}\\
operator (\texttt{+}) & scalars or arrays & same as intrinsic\\
operator (\texttt{-}) & scalars or arrays & same as intrinsic\\
operator (\texttt{*}) & scalars or arrays & same as intrinsic\\
operator (\texttt{/}) & scalars or arrays & same as intrinsic\\
operator (\texttt{**}) & scalars & same as intrinsic\\
\texttt{sum} & 1D array & same as intrinsic\\
\texttt{dot\_product} & 1D arrays & same as intrinsic\\
\texttt{log\_sum\_exp} & 1D array & For input \texttt{x}, calculate \texttt{log(sum(exp((x))))}\\
\texttt{matmul} & 2D arrays & same as intrinsic\\
\end{tabular}
\end{center}

\bibliographystyle{plain}
\bibliography{ref}
\end{document}
